% options:
% thesis=B bachelor's thesis
% thesis=M master's thesis
% czech thesis in Czech language
% slovak thesis in Slovak language
% english thesis in English language
% hidelinks remove colour boxes around hyperlinks

\documentclass[thesis=B,czech]{FITthesis}[2012/06/26]

\usepackage[utf8]{inputenc} % LaTeX source encoded as UTF-8

\usepackage{graphicx} %graphics files inclusion
% \usepackage{amsmath} %advanced maths
% \usepackage{amssymb} %additional math symbols

%\usepackage{stdpage} % nastaveni normostrany
\usepackage{dirtree} %directory tree visualisation

\usepackage{hyphenat}

% % list of acronyms
% \usepackage[acronym,nonumberlist,toc,numberedsection=autolabel]{glossaries}
% \iflanguage{czech}{\renewcommand*{\acronymname}{Seznam pou{\v z}it{\' y}ch zkratek}}{}
% \makeglossaries

\newcommand{\tg}{\mathop{\mathrm{tg}}} %cesky tangens
\newcommand{\cotg}{\mathop{\mathrm{cotg}}} %cesky cotangens

% % % % % % % % % % % % % % % % % % % % % % % % % % % % % %
% ODTUD DAL VSE ZMENTE
% % % % % % % % % % % % % % % % % % % % % % % % % % % % % %

\department{Katedra aplikované matematiky}
\title{Forenzní analýza v IT}
\authorGN{Ladislav} %(křestní) jméno (jména) autora
\authorFN{Martínek} %příjmení autora
\authorWithDegrees{Bc. Ladislav Martínek} %jméno autora včetně současných akademických titulů

\abstractCS
{
Tato práce se zabývá problematikou digitální forenzní analýzy webových prohlížečů. V prvních
kapitolách je popsán proces forenzní analýzy, používané metodologie a postupy a také problematika
analýz webových prohlížečů. Zejména pak typy informací, které prohlížeč ukládá a problémy s
bezpečností. Na základě teoretických základů byl pak vytvořen teoretický a praktický postup získání
dat z webového prohlížeče Mozilla Firefox. Praktický postup využívá běžné linuxové nástroje a
vlastní aplikaci, která byla vytvořena v rámci této práce. Na základě navrženého praktického postupu
byla provedena forenzní analýza prohlížeče.


}

\abstractEN
{
  Forenzní analýza v IT
}


\placeForDeclarationOfAuthenticity{TODO}

\keywordsCS
{
  Forenzní analýza v IT, digitaní forenzní analýza , informační bezpečnost
}
\keywordsEN
{
  Forensic analysis in IT, digital forensic analysis, information security
}

\begin{document}

% \newacronym{CVUT}{{\v C}VUT}{{\v C}esk{\' e} vysok{\' e} u{\v c}en{\' i} technick{\' e} v Praze}
% \newacronym{FIT}{FIT}{Fakulta informa{\v c}n{\' i}ch technologi{\' i}}

\begin{introduction}
  \label{sec:uvod}

V dnešní době je s digitálními zařízenímy spojena většina každodenních činností, ať už jsou to počítače, mobilní telefony, ale v dnešní době i hodinky nebo žárovky, které například ovládáme přes svůj mobilní telefon, a také další domácí spotřebiče. Můžeme například platit behotovostně ať už pomocí karty nebo i chytrých hodinek. Všechny tyto digitální technologie jsou v našem životě zastoupeny čím dál větší měrou a nahrazují staré postupy. 

Všechna tyto digitální zařízení se však mohou stát terčem útoku. Útoky na digitální systémy se stavájí každodenní rutinou a je tedy nutné umět takové útoky odhalit, bránit se proti nim. Počet útoků na tyto systémy narůstá každým rokem. Nemusí se přímo jednat jen o útoky proti počítačům, ale například i zneužití digitální techniky k páchání trestného činu. Například počítače v bytě, kde proběhla vražda.

Pokud se nějaký fyzická trestný čin, bude místo činu zkoumáno oprávněnými osobami. Cílem ohledání místa činu je rekonstrukce události a nalezení viníka. Důležitou součástí procesu je sběr důkazů, které představují třeba otisky prstů. Jinak tomu není ani v případě digitálních systému, kde je nutné zajistit důkazy například data po škodlivém softwaru. Tyto data jsou velice nestálá a je nutné s nimi náležitě pracovat.

Vědou, která se tímto zabývá je forenzní analýza digitálních dat (angl. digital forensics). V této práci tuto oblast popíši a pokusíme se příblížit některé metody, nástroje, postupy, ale také stručnou histrorii. Vágně řečené jde v této oblasti o analýzu dat s cílem zjistit co, kdy, kde a jak se stalo.
 

\end{introduction}

\chapter{Forenzní analýza}
Forenzní analýza je pojem pro analýzu a vyšetřování, které se prování za účelem zdokumentování události (nejčastěji v oblasti bezpečnosti). Dále zjištění jejich důvodů, průběhu a hledání viníků a objektivních důkazů, které by případné viníky mohli usvědčit (u soudu)\cite{for_uvod}.

V knize \cite{RECHAndBook} je digitální forenzní analýza definována jako soubor technik a nástrojů používaných pro hledání důkazů na počítači, které mohou být použity v uživatelův neprospěch. Důkazy nemusíjí přímo souviset s počítačovou kriminalitou. Digitalní forenzní analýzu lze aplikovat napčíklad pří vyšetřovní trestné činnosti, vydírání, krádeží, podvodů a padělání.

Forenzní analýza v informačních technologiích neboli také forenzní analýza digitální dat (pokud bude dále v textu zmíněna forenzní analýza půjde už pouze jen konkrétně o forenzní analýzu digitální dat) se zabývá výše zmíněnými problémy s využitím digitální dat, jak je popsáno v \cite{for_root}. Oproti definici výše se zde zmiňují téměř libovolná digitální data. Tato digitální data jsou pak většinou uložena na disku v počítači, ale obecně to může být libovolné digitální médium.

Forenzní analýza se využívá napříč velkým množstvím oborů, například v kriminalistice, interním vyšetřovaní ve firmě, ale může být využita například i při obnově dat nebo občanskoprávním řízení.

Forenzní věda byla vytvořena za účelem řešení specifických potřeb donucovacích orgánů, aby se co nejlépe využila tato nová forma elektronických důkazů, které mohou dopomoci objasntit a vyřešit daný případ. Postupy v digitalní forenzní analýze jsou řízeny směrnice a postupy, které je nutné dodržovat.

Jak se píše v \cite{for_uvod}, počítačová forenzní věda je ve svém jádru odlišná od většiny tradičních forenzních disciplín. Počítačový materiál, který je zkoumán, a techniky, které má provádějící technik k dispozici, jsou většinou produkty soukromého sektoru. Kromě toho, na rozdíl od tradičních forenzních analýz, je běžně vyžadováno provádění počítačových zkoušek prakticky na jakémkoli fyzickém místě, nejen v kontrolovaném laboratorním prostředí. Obecně není datová forenzní analýza prostorově náročná, ale může být náchylná na drobné nepřesnosti, kterým je nutné se vyvarovat, aby výsledky mohli být použitelné. Digitální forenzí analýza většinou nevýtváří přímé závěry, ale poskytuje příme informace, kterými bývají závěry podloženy.


\section{Kroky digitální forenzní analýzy}
Mezi hlavní kroky forenzní analýzi podle \cite{carroll2008computer} tedy můžeme považovat kroky: zabavení, získávání, validace a identifikace, analýza a vykazování výsledků, které potom podrobněji popíši v nasledujících sekcích.

\subsection{Zabavení}
Krok zabavení zahrnuje označení prvků, které budou použity v pozdějších procesech. Jsou pořizovány fotografie scény a poznámky. Například počítač je nutné zapečetit, aby nemohlo být napadeno to, že byl obsah modifikován. V \cite{for_sez} zmiňuje zajímavou otázku, na kterou je třeba v tuto fázi odpovědět: Vytáhnout zástrčku v síti nebo ne. Ponechání systému online během pokračování může útočníka upozornit, což mu umožní vymazat stopy útoku a zničit důkazy. Útočník může také nechat přepínač, který zničí důkazy, jakmile se systém přepne do režimu offline. Za takových okolností může být nutné nebo vhodné shromáždit důkazy ze systému, když je spuštěn. Pro přijetí jakéhokoliv postupu je nutné kroky vysvětlit. Cílem fáze je zajištění a uchování důkazů.

\subsection{Získávání}
Po první fázi je přistopeno k získávání dat. Součástí této fáze je většinou i fáze duplikace, kdy je kopie dat bezpečně uložena. Data musí být získána beze změny nebo poškození zdroje, který má být analyzován později. Nezákonné zabavení nebo nevhodný postup může ovlivnit platnosti důkazů u soudu. Z tohoto důvodu by metody získávání důkazů měly být forenzně spolehlivé a ověřitelné. Získavání může být fyzická nebo logická. Při fyzickém získávání je bitový obraz snímán z fyzického paměťového média, zatímco v logickém získávání je řídký nebo logický obraz zachycen z paměťového média. V obou případech je třeba použít writeblockery, aby se zabránilo úpravě důkazů. Podle \cite{for_sez} se vždy doporučuje zahájit snímání od nejmenších po nejmenší data. Pořádí důležitosti je pak: 

\begin{enumerate}
\item Registry, cache
\item Stav sítě (mezipaměť ARP a směrovací tabulka)
\item Běžící procesy
\item Moduly a statistiky jádra
\item Hlavní paměť
\item Dočasné soubory na disku
\end{enumerate}


\subsection{Validace a identifikace}

Data jsou tedy většinou duplikována na jiné médium. Duplicitní obrázek musí být ověřen jestli je totožný se zdrojem, porovnáním hodnoty hash získaného obrazu nebo kopie a původních dat médií nebo kontrolním součtem.


Podle \cite{carroll2008computer} je postup následující: Technici opakují proces identifikace pro každou položku v seznamu extrahovaných dat. Nejprve určují, o jaký typ položky jde. Pokud pro forenzní žádost není relevantní, jednoduše ji označí jako zpracovanou a přesunou se dále. Stejně jako ve fyzickém vyhledávání, pokud technik narazí na věc, která je usvědčující, ale mimo rozsah původního příkazu k prohlídce, doporučuje se, aby examinátor okamžitě zastavil veškerou činnost, informoval příslušné osoby, včetně žadatele, a počkal pro další pokyny. Například orgány činné v trestním řízení by mohly zabavit počítač jako důkaz o daňových podvodech, ale zkoušející může najít obrázek z jiné trestné čínnosti. Důležité je se pokusit rožšířit pravomoce pro další práci.

Dále také technik může data, která ukazují na zcela nový potenciální zdroj dat. Například mohou najít nový e-mailový účet. V tomto okamžiku je vhodné, aby zkoušející informovali žadatele o svých počátečních zjištěních.

\subsection{Analýza}
Ve fázi analýzy připojí technik všechny zjištěné informace a vykreslí kompletní obrázek pro žadatele\cite{carroll2008computer}. U každé položky v seznamu relevantních údajů technici odpovídají na otázky zmíněné v úvodu a to co, kdy, kde, kdo a jak. Je zde snaha o co nejpodrobnější vysvětlení.

Zkoušející často dokážou vytvořit nejcennější analýzu tím, že se podívají na to, kdy se něco stalo, a vytvářejí časovou osu, která vypráví souvislý příběh \cite{carroll2008computer}. U káždého souboru nebo informace je snaha o zasazení na osu tedy o zjištění, kdy byla vytvořena, zpřístupněna, změněna, přijata, odeslána, zobrazena, odstraněna a spuštěna. 

Během tohoto postupu by měly být dodržovány vhodné metodiky a standardy (popsané v následující části)\cite{for_sez}. Vyšetřovatel by měl prozkoumat získanou kopii nebo obraz média, nikdy ne originálního média. Na rozdíl od fáze zabavení a identifikace vyžaduje fáze analýzy odborníky. Technici musí mít řádné znalosti a musí být řádně vyškoleni. 


\uv{\textit{K vyhledávání konkrétních důkazů se používají dva druhy analýz. První je analýza fyzická, která má za úkol najít například nějaký řetězec z obsahu disku a to v rámci všech sektorů disku, který berete jako celek. Logická analýza spočívá v analýze jednotlivých souborů.  Mezi Logickou patří například soubory v souborovém systému.}}\cite{for_root} Podle toho se tato analýza se bude lišit pro různé souborové systémy (Windows, Linux).

\subsection{Vykazování výsledků}
Na závěr je nutné vytvořit zprávu s výsledky. Ve zprávě musí být popsáný všechny kroky provedené během šetření. Obsahuje zprávy o předchozích krocích a především získaní dat, jejich ověření a analýzu. Je nutné uvést všechny nástroje a hardwarové prvky. Celý postup musí být opakovatelný nezávislou osobou a také musí být ověřitelná nezaujatost, nestrannost a nezávislost. Výsledky musí jasně prezentovatelné i osobám mimo daný obor.

\section{Typy forenzní analýzy}
Dále můžeme forenzní analýzu rozdělit na nasledující typy tak jak je uvedeno v \cite{for_types}. V jiné literatuře je možné narazit i na jiné rozdělení, které je bývají většinou stručnější.

\subsection{Computer Forensics}
the identification, preservation, collection, analysis and reporting on evidence found on computers, laptops and storage media in support of investigations and legal proceedings.
\subsection{Network Forensics}
the monitoring, capture, storing and analysis of network activities or events in order to discover the source of security attacks, intrusions or other problem incidents, i.e. worms, virus or malware attacks, abnormal network traffic and security breaches.
\subsection{Mobile Devices Forensics}
the recovery of electronic evidence from mobile phones, smartphones, SIM cards, PDAs, GPS devices, tablets and game consoles.
\subsection{Digital Image Forensics }
the extraction and analysis of digitally acquired photographic images to validate their authenticity by recovering the metadata of the image file to ascertain its history.
\subsection{Digital Video/Audio Forensics}
the collection, analysis and evaluation of sound and video recordings. The science is the establishment of authenticity as to whether a recording is original and whether it has been tampered with, either maliciously or accidentally.
\subsection{Memory forensics}
the recovery of evidence from the RAM of a running computer, also called live acquisition.

\section{Předmět forenzní analýzy}

Předmětem forenzní analýzyt jsou
Nejčastěji jsou tato data nacházena na záznamových mediích v podobě
souborů (platná data), smazaných souborů (neplatná data), v podobě fragmentů
dat (částečně přepsané soubory), dále pak v podobě dat dočasně uložených
v operační paměti nebo v podobě záznamů běžících služeb (logů).
Aby analýza digitálních dat mohla být považována za forenzní, musí splňovat
obecné podmínky (zásady) forenzního zkoumání. (viz. kapitola Zásady digitální
forenzní analýzy)

\section{Techniky}

wikipedie a neco

\section{Použití jako důkaz}

\section{Omezení a rizika}


Tato kapitola je zaměřena na popis možných ohrožení a rizik v celém procesu práce
s digitálními stopami. Vznik rizik a ohrožení nelze při práci s digitálními stopami nikdy
vyloučit. I při důsledném dodržení všech existujících bezpečnostních pravidel, zásad
a doporučených postupů, mohou nastat situace, při kterých vzniknou rizika, nebo ohrožení
digitálních stop. Tyto rizika mohou být způsobeny technikou, nebo použitou technologií,
například při závadách, poruchách, nebo při selhání zajištěných zařízení, které obsahují
digitální stopy, nebo technologických datových médií, na kterých jsou tyto stopy uloženy.
Za dalšími riziky a ohroženími stojí většinou selhání lidského faktoru, nedodržení pravidel
pro bezpečnou manipulaci s digitálními stopami, nebo nedodržení zásad pro jejich
skladování. U rizik způsobených selháním lidského faktoru se může jednat o rizika
způsobená úmyslně, neúmyslně, nebo způsobená nedbalostí. V následujícím textu budou
podrobněji popsána nejdůležitější rizika a ohrožení v jednotlivých etapách práce
s digitálními stopami a zároveň také navrhovaná protiopatření, kterými se uvedená rizika
zmírní, nebo úplně eliminují.
\cite{for_baka}

V průběhu digitální forenzní analýzy nelze vyloučit rizika související s digitálními stopami.
Může se jednat o problémy s technikou, selhání zařízení obsahující digitální stoupu. Ale
nejčastěji se bude jednat o riziko lidského faktoru, tedy nedodržování předepsaných pravidel
a postupů. Toto ohrožení může být způsobeno neúmyslně, ale taky s úmyslem zmanipulovat
vyšetřování. Ladislav Vyskočil ve své práci uvádí především tato rizika:
• Nezajištění všech digitálních stop
• Neodborné zajištění digitálních stop
• Nesprávné zabalení a ověření digitálních stop
• Nesprávná, nebo neúplná dokumentace
• Znehodnocení zajištěných stop
• Úmyslné zničení dat
• Nemožnost rozšifrovat data
• a další…
Většině rizik spojených s digitální stopou lze předcházet dodržováním postupů a doporučení
a obecných bezpečnostních pravidel.

\section{Právní aspekty}


\chapter{Nástroje digitální forenzní analýzy}

Ve forenzní analýze je používáno mnoho nástrojů, který proces usnadňují a pomáhají předejít problémům. Tyto nástroje můžeme rozdělit mezi hardwarové (vetšinou fyziká krabička) nebo sofwarové (program).

Dále tu na každý takovýto nástroj máme podle \cite{for_need} následující požadavky:

\begin{itemize}
\item Usability - 
\item Comprehensive - 
\item Accuracy - 
\item Deterministic - 
\item Verifiable - 
\end{itemize}

Další doporučení pro nástroje je read-only (původní data vůbec nemodifikuju. Data mohou být snadno zkopírovatelná, sice toto není nutnost, ale pro budoucí verifikaci výsledků je to žádoucí a kopie dat bude vyžadována. 


 Usability: Present data a layer of abstraction that is useful
to an investigator (Complexity Problem)

Comprehensive: Present all data to investigator so that both
Inculpatory and Exculpatory Evidence can be identified

 Accuracy: Tool output must be able to be verified and a
margin of error must be given (Error Problem)

 Deterministic: A tool must produce the same output when
given the same rule set and input data.

 Verifiable: To ensure accuracy, one must be able to verify
the output by having access to the layer inputs and outputs.

 Verification can be done by hand or a second tool set.



\section{Hardware}
duprikatory
write blokery


\section{Software}

Téměř veškeré postupu forenzní analýzy jsou stadardizovány a pro provádění forenzní analýzy existuje mnoho programů, které jsou vytvořeny podle těchto standardů. Nástroje pomáhají se vyvarovat chybám, které by mohli výsledky snehodnotit tak, že by výsledky forenzní analýzy nebyly použitelné u soudu. Informace o existujícím softwaru jsem čerpal z \cite{for_soft}.

První nástroj je od firmy AccessData. Forensic Toolkit FTK 7 je uznáván jako světový standard pro forenzní analýzu. Tato platforma představuje špičku v oboru analýzy digitálních dat, dešifrování, lámání hesel, a to všechno v intuitivním a přizpůsobitelném rozhraní. Nástroj také umožnuje pracovat s velkými objemy dat, díky distribuované databázové struktuře.

Dálším nástrojem je například Belkasoft Evidence Center 2019 od firmy Belkasoft. Program umožňuje forenznímu expertovi snadno prohledávat, analyzovat, uchovávat a sdílet digitální důkazy zajištěné na paměťových médiích nebo v operační paměti počítače. Nástroj dokáže zajistit důkazy z mnoha zdrojů, jako jsou pevné disky, obrzay paměti, zálohy iOS, Androidu či BlackBerry, nebo obrazy z UFED, JTAG nebo chip-off analýzy \cite{for_soft}.

Dalším nástrojem je třeba EnCase Forensic 8 od Guidance Software/OpenText. Nástroje nejsou levné, ale když vezmeme v úvahu, že pracujeme s daty, která tuto hodnotu mnohdy i převýší, tak se vyplatí investice do nástroje, který pomůže předejít chybám a usnadní digitální forenzní analýzu.

\chapter{Anti-forenzní analýza}

Anti-forenzní analýza je soubor technik používaných jako protiopatření k digitální forenzní analýze. V práci jsem se rozhodl toto zmínit, protože znalost těchto technik je určitě duležitá i pro forenzní techniky a naopak. Podle \cite{anti-prezi} jde rozdělit do následujících kategorií.

\begin{itemize}

\item[Schovávání dat] - Nemusí se jednat vždy přímo o snahu, aby data nebyla viděť nebo nebyla na místě kde jsou hledána. Typicky skryté oddíly disku, chybné sektory, skryté složky. Další možností, kdy data jsou viditelná ale nečitelná je šifrování. Zašifrování dat dostatečně silným klíčem je téměř neprolomitelné. Dálší možností může být napřiklad steganografie. Jedná se o ukrytí zprávy. Například využítím nějakého kanálu v obrázku.
\item[Vymazání artefaktů] - Zde je jdná například o důkladné smazání prázdného disku. Klasické odstranění souboru nevymaže data z disku fyzicky. 
\item[Nástrahy a zmatení] - Kde jsou data uložena. Trojské koně, zombii účty nebo například umyslná záměna inforací. 
\item[Útoky proti nástrojům pro forenzní analýzu] - Ja je zmíněno v \ref{anti-prezi}. Toto je relativně nová možnost. Je to zapříčeněno standardizací forenzní analýzy, kdy je znám přesný postup, který je prováděn softwarem. Závislost na forenzním softwaru velmi ohrožuje zkoumaná data, forenzní software může být napaden a ovlivněn tak aby stopu zahladil napřílad. 

\end{itemize}



\begin{conclusion}
Co jsem jak pospal

Forenzní analýza digitálních dat spojuje velmi široký okruh témat. K provádění datové forenzní anaýzy je nutné mít znalosti ze sítového provozu, hardwaru a jeho chovaní na okolní podněty. Dále také velké škály programů. Bez daných znalostí nejsme schopni provést analýzu tak aby byl výsledek věrohodný a použitelný. V nejhorším případě by to vedlo ke zničení důkazů. 

V mé práci jsem popsal vědní obo digitální forenzní analýzy. Tento obor je velice obsháhlý a je možné ho dělit na plno částí, například podle popsaných kapitol. O každé kapitole by mohla být napsána přinejmenších stejná práce. Pro více informací doporučuji citovanou literaturu.






\end{conclusion}

\bibliographystyle{csn690}
\bibliography{biblio}

\appendix

\chapter{Seznam použitých zkratek}
% \printglossaries
\begin{description}

    \item[IT] Information technology


\end{description}

\end{document}
